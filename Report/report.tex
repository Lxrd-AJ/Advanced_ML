\documentclass[10pt,twocolumn]{article} 
\usepackage{simpleConference}
\usepackage{times}
\usepackage{graphicx}
\usepackage{amssymb}
\usepackage{url,hyperref}

\begin{document}

\title{Convolutional Neural Networks for Satellite Image Segmentation}

\author{Ganiyu Ibraheem, Philipp Seybold, James Purcell, Ahmed Abouseida and Evgeny Saveliev. \\
\\
Electronics and Computer Science \\
University of Southampton \\
\today
}


\maketitle
\thispagestyle{empty}

\begin{abstract}
This report highlights how a convolutional neural network was used for segmenting objects in satellite images. This project is inspired by the \href{https://www.kaggle.com/c/dstl-satellite-imagery-feature-detection}{Kaggle DSTL satellite imagery feature detection challenge} where the requirement was to build a learning machine that identify features in overhead imagery to alleviate the burdens on their image analysts.

Our solution is based on U-Net, a convolutionary neural network which we trained with satellite images of 19-spectral bands and achieved a \underline{90\%} classification accuracy. %TODO: Update classification accuracy
%TODO: Summarise report here
\end{abstract}


\section{Introduction}
%TODO: Introduction needed

	\subsection{Related Work}
	%TODO: Subsection


\section{Satellite Imagery Dataset}
%TODO: Talk about the dataset here, the format it was in, how we transformed it and collated them for preprocessing
	\subsection{Image Annotation}
	%TODO: Talk about how the dataset was annotated
	\subsection{Imagery statistics}
	%TODO: Talk about the number of classes in the dataset, where they balanced? How many for each class etc
		\begin{description}
		  \item[$\bullet$] Segmentation Quality: %TODO: what was the quality of the segmentation in the ground truth
		  \item[$\bullet$] Object Naming: %TODO: Where there any ambiguities in the class labels e.g car and large trucks are ambigious; what did we do about that
		\end{description}

\section{U-Net Convolution Network}
%TODO: Talk about U-Net here, 
%TODO: Include Architecture diagram for U-Net here
%TODO: Explain some basic deep learning terms e.g convolution, deconvolution etc

\section{Experiments}
%TODO: Talk about experiments done here
	\subsection{Metrics}
		\begin{description}
			\item[$\bullet$] Pixel Accuracy: indicates the proportion of correctly classified pixels
			\item[$\bullet$] Mean Accuracy: indicates the proportion of correctly classified pixels averaged over all the classes.
			\item[$\bullet$] Mean IoU: indicates the intersection-over-union between the predicted and ground-truth pixels, averaged over all the classes.
			\item[$\bullet$] Weighted IoU: indicates the IoU weighted by the total pixel ratio of each class.
		\end{description}


\section{Conclusion}
%TODO: Conclusion

\section{Future Work}

This material is important -- part of the value of a paper is showing how the work sets new research directions. I like bullet lists here. A couple of things to keep in mind:
\begin{description}
 	\item[$\bullet$]  We are currently extending the algorithm to use Mask-RCNN and PSPNet, and preliminary results are encouraging.
	\item[$\bullet$] We are exploring hand crafting a more extensive dataset using Google Earth; Gathering images at different intervals of heights above sea level e.g 200m, 500m, 1km etc and utilising a site like Amazon Turk or \href{https://appen.com/}{Appen}
\end{description}


\bibliographystyle{abbrv}
\bibliography{refs}
\end{document}
